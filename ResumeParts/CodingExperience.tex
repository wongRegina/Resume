%%%%%%%%%%%%%%%%%%%%%%%%%%%%%%%%%%%%%%
%     CODING EXPERIENCE
%%%%%%%%%%%%%%%%%%%%%%%%%%%%%%%%%%%%%%
\section{Projects}

\runsubsection{GoLogoLo}
\descript{| Fundamentals of Software Development \hspace{1.3 in} Stony Brook, NY | Spring '20}
\begin{tightemize}
\item Developed a full-stack MERN application complete with a back-end and database using MongoDB and GraphQL
\item Employed Passport.js to create a user authentication system for the logo maker
\end{tightemize}
\vspace{0.1 cm}

\runsubsection{ PCAP Analysis}
\descript{| Computer Networks \hspace{2.6 in} Stony Brook, NY | Spring '20}
\begin{tightemize}
\item Analyzed a PCAP file to characterize the TCP flows in the trace (version of TCPDump)
\item Performed a byte-level programming to read a PCAP file and extracted the ARP header element
\end{tightemize}
\vspace{0.1 cm}

\runsubsection{Quadrilaterals in Python}
\descript{| Programming Abstraction \hspace{1.2in} Stony Brook, NY | Fall ‘19}
\begin{tightemize}
\item Created three classes for different quadrilaterals and a sorter class (sorted the shapes by x-coordinate)
\item Employed Python unit testing for all methods
\end{tightemize}
\vspace{0.1 cm}

% \runsubsection{Hangman}
% \descript{| Programming Abstraction \hspace{2.7in} Stony Brook, NY | Fall ‘19}
% \begin{tightemize}
% \item Created a Hangman Game that chose a random word from a text file and took in keyboard input
% \item Implemented modular programming in JavaFX
% \end{tightemize}
% \vspace{0.1 cm}

% \runsubsection{A Healthy Game}
% \descript{| HackHealth \hspace{3.1in} Stony Brook, NY | Spring ‘19}
% \begin{tightemize}
% \item Developed a game that tests knowledge on nutrition with the use of JavaFX and a 2D-array
% \item  Learned JavaFX during this hackathon
% \end{tightemize}
% \sectionsep

% \runsubsection{Bash Terminal}
% \descript{| Data Structures \hspace{2.95 in} Stony Brook, NY | Fall ‘18}
% \begin{tightemize}
% \item Designed a Bash terminal that would go through a tree, that can have a max of 10 children
% \item Search for an item in the tree with the use of recursion (depth-first traversal)
% \end{tightemize}